\documentclass[12pt, a4paper]{article}

\usepackage{cmap}					
\usepackage[T2A]{fontenc}		
\usepackage[utf8]{inputenc}			
\usepackage[english,russian]{babel}
\usepackage{amsmath}
\usepackage{latexsym}
\usepackage{tikz}
\usetikzlibrary{arrows}

\begin{document}

%\documentclass[12pt, a4paper]{article}
%\usepackage{tikz}
%\usepackage{cmap} 
%\usepackage[T2A]{fontenc} 
%\usepackage[utf8]{inputenc} 
%\usepackage[english,russian]{babel}
%\usepackage{amsmath}
%\usepackage[left=2cm,right=2cm,
%    top=2cm,bottom=2cm,bindingoffset=0cm]{geometry}
%\title{Алгоритмы кластеризации}
%\date{}
%\begin{document}
%\maketitle

\section*{K-means}
Алгоритм разбивает множество элементов на известное число кластеров. Основной идеей алгоритма является минимизировать среднее расстояние между каждым из объектов в кластере и его центроидом. 

\begin{itemize}

\item \( \varOmega \) = \{ \( \omega_1, \omega_2, \ldots, \omega_k \} \) -- множество кластеров.

\item \( X = \{ x_1, x_2, \ldots , x_n \} \) -- множество объектов. 

\item \( X(x_1, x_2) = \sqrt{(x_{1_1} - x_{2_1})^2 + (x_{1_2} - x_{2_2})^2 + \ldots + (x_{1_n} - x_{2_n})^2 } \) -- растояние между объектами.

\item \( \mu(\omega) = \frac{1}{|\omega|} \sum_{x \in \omega}x \) -- центроид кластера, где \( |\omega| \) -- мощность множества кластера.


\end{itemize}

\[
\begin{tikzpicture}
    \begin{scope}[yshift=-7.5cm, xshift=7.5cm]
        \draw (-3.5, 3.5 ) -- (3.5, 3.5);
        \draw (-3.5, -3.5) -- (3.5, -3.5);
        \draw (-3.5, -3.5) -- (-3.5, 3.5);
        \draw (3.5, 3.5) -- (3.5, -3.5);

        \draw[blue, fill=blue] (-1.5, -2.5) circle (0.2);
        \draw[blue, fill=blue] (-1.75, -0.5) circle (0.2);
        \draw[blue, fill=blue] (-1.5, 1.4) circle (0.2);
        \draw[blue, fill=blue] (-2, 0.75) circle (0.2);
        \draw[blue, fill=blue] (-2.625, 1.8) circle (0.2);
        \draw[blue, fill=blue] (1.75, -3.2) circle (0.2);
        \draw[blue, fill=blue] (3, -2.8) circle (0.2);
        \draw[blue, fill=blue] (3, -0.5) circle (0.2);
        \draw[blue, fill=blue] (1.75, -0.5) circle (0.2);
        \draw[blue, fill=blue] (2.7, 1.2) circle (0.2);
        \draw[blue, fill=blue] (2.5, 1.8) circle (0.2);
        \draw[blue, fill=blue] (0.5, 1) circle (0.2);
        \draw[blue, fill=blue] (-0.4, 2.3) circle (0.2);
        \draw[blue, fill=blue] (-0.9, 2.9) circle (0.2); 
        \draw[black] (-3, 1.5) -- (-2.5, 1);
        \draw[black] (-3, 1) -- (-2.5, 1.5);
        \draw[black] (0.25, 3) -- (0.75, 3.5);
        \draw[black] (0.25, 3.5) -- (0.75, 3);
        \filldraw[black](0.5, 3.25) circle (2pt);
        \filldraw[black] (-2.75 , 1.25) circle (2pt);
    \end{scope}
\end{tikzpicture}
\]

\begin{enumerate}

\item На первом шаге фиксируется начальное приближение. 


\[
\begin{tikzpicture}
    \begin{scope}[xshift=7.5cm, yshift=7.5cm]
        \draw (-3.5, 3.5 ) -- (3.5, 3.5);
        \draw (-3.5, -3.5) -- (3.5, -3.5);
        \draw (-3.5, -3.5) -- (-3.5, 3.5);
        \draw (3.5, 3.5) -- (3.5, -3.5);

        \draw[red, fill=red] (-1.5, -2.5) circle (0.2);
        \draw[red, fill=red] (-1.75, -0.5) circle (0.2);
        \draw[red, fill=red] (-1.5, 1.4) circle (0.2);
        \draw[red, fill=red] (-2, 0.75) circle (0.2);
        \draw[red, fill=red] (-2.625, 1.8) circle (0.2);
        \draw[yellow, fill=yellow] (1.75, -3.2) circle (0.2);
        \draw[yellow, fill=yellow] (3, -2.8) circle (0.2);
        \draw[yellow, fill=yellow] (3, -0.5) circle (0.2);
        \draw[yellow, fill=yellow] (1.75, -0.5) circle (0.2);
        \draw[yellow, fill=yellow] (2.7, 1.2) circle (0.2);
        \draw[yellow, fill=yellow] (2.5, 1.8) circle (0.2);
        \draw[yellow, fill=yellow] (0.5, 1) circle (0.2);
        \draw[yellow, fill=yellow] (-0.4, 2.3) circle (0.2);
        \draw[yellow, fill=yellow] (-0.9, 2.9) circle (0.2); 
        \draw[black] (-3, 1.5) -- (-2.5, 1);
        \draw[black] (-3, 1) -- (-2.5, 1.5);
        \draw[black] (0.25, 3) -- (0.75, 3.5);
        \draw[black] (0.25, 3.5) -- (0.75, 3);
        \draw[black] (1.5, -3.4) -- (-1.5, 3.4);
        \filldraw[black](0.5, 3.25) circle (2pt);
        \filldraw[black] (-2.75 , 1.25) circle (2pt);

        
    \end{scope}
\end{tikzpicture}
\]

\item На втором шаге объекты разбиваются на кластеры, при условии минимизации расстояния от объектов до центроида. 

\[
\begin{tikzpicture}
    \begin{scope}[xshift=7.5cm, yshift=7.5cm]
        \draw (-3.5, 3.5 ) -- (3.5, 3.5);
        \draw (-3.5, -3.5) -- (3.5, -3.5);
        \draw (-3.5, -3.5) -- (-3.5, 3.5);
        \draw (3.5, 3.5) -- (3.5, -3.5);

        \draw[red, fill=red] (-1.5, -2.5) circle (0.2);
        \draw[red, fill=red] (-1.75, -0.5) circle (0.2);
        \draw[red, fill=red] (-1.5, 1.4) circle (0.2);
        \draw[red, fill=red] (-2, 0.75) circle (0.2);
        \draw[red, fill=red] (-2.625, 1.8) circle (0.2);
        \draw[yellow, fill=yellow] (1.75, -3.2) circle (0.2);
        \draw[yellow, fill=yellow] (3, -2.8) circle (0.2);
        \draw[yellow, fill=yellow] (3, -0.5) circle (0.2);
        \draw[yellow, fill=yellow] (1.75, -0.5) circle (0.2);
        \draw[yellow, fill=yellow] (2.7, 1.2) circle (0.2);
        \draw[yellow, fill=yellow] (2.5, 1.8) circle (0.2);
        \draw[yellow, fill=yellow] (0.5, 1) circle (0.2);
        \draw[yellow, fill=yellow] (-0.4, 2.3) circle (0.2);
        \draw[yellow, fill=yellow] (-0.9, 2.9) circle (0.2); 
        \draw[black] (1.5, -3.4) -- (-1.5, 3.4);
        \draw[->][black] (-2.75, 0.9) -- (-2.3, 0.1);
        \draw[->][black] (0.5, 2.9) -- (1.2, 0.75);
        \draw[black] (-3, 1.5) -- (-2.5, 1);
        \draw[black] (-3, 1) -- (-2.5, 1.5);
        \draw[black] (0.25, 3) -- (0.75, 3.5);
        \draw[black] (0.25, 3.5) -- (0.75, 3);
        \filldraw[black](0.5, 3.25) circle (2pt);
        \filldraw[black] (-2.75 , 1.25) circle (2pt);
        \draw[black] (-2.5, -0.5) -- (-2, 0);
        \draw[black] (-2, -0.5) -- (-2.5, 0);
        \filldraw[black](-2.25, -0.25) circle (2pt);
        
        \draw[black] (1, 0.2) -- (1.5, 0.7);
        \draw[black] (1.5, 0.2) -- (1, 0.7);
        \filldraw[black](1.25, 0.45) circle (2pt);
        
    \end{scope}
\end{tikzpicture}
\]
\item Пересчитываются значения центроидов.

\[
\begin{tikzpicture}
        \begin{scope}
        \draw (-3.5, 3.5 ) -- (3.5, 3.5);
        \draw (-3.5, -3.5) -- (3.5, -3.5);
        \draw (-3.5, -3.5) -- (-3.5, 3.5);
        \draw (3.5, 3.5) -- (3.5, -3.5);
    
        \draw[red, fill=red] (-1.5, -2.5) circle (0.2);
        \draw[red, fill=red] (-1.75, -0.5) circle (0.2);
        \draw[red, fill=red] (-1.5, 1.4) circle (0.2);
        \draw[red, fill=red] (-2, 0.75) circle (0.2);
        \draw[red, fill=red] (-2.625, 1.8) circle (0.2);
        \draw[yellow, fill=yellow] (1.75, -3.2) circle (0.2);
        \draw[yellow, fill=yellow] (3, -2.8) circle (0.2);
        \draw[yellow, fill=yellow] (3, -0.5) circle (0.2);
        \draw[yellow, fill=yellow] (1.75, -0.5) circle (0.2);
        \draw[yellow, fill=yellow] (2.7, 1.2) circle (0.2);
        \draw[yellow, fill=yellow] (2.5, 1.8) circle (0.2);
        \draw[yellow, fill=yellow] (0.5, 1) circle (0.2);
        \draw[red, fill=red] (-0.4, 2.3) circle (0.2);
        \draw[red, fill=red] (-0.9, 2.9) circle (0.2); 
 

        \draw[black] (0.5, 3.4) -- (-1.5, -3.4);
        \draw[->][black] (-2.25, 0) -- (-1.8, 1);
        \draw[->][black] (1.3, 0.1) -- (1.9, -0.4);
        \draw[black] (-2.5, -0.5) -- (-2, 0);
        \draw[black] (-2, -0.5) -- (-2.5, 0);
        \filldraw[black](-2.25, -0.25) circle (2pt);
        \draw[black] (1, 0.2) -- (1.5, 0.7);
        \draw[black] (1.5, 0.2) -- (1, 0.7);
        \filldraw[black](1.25, 0.45) circle (2pt);
        

        \draw[black] (-2, 1) -- (-1.5, 1.5);
        \draw[black] (-1.5, 1) -- (-2, 1.5);
        \filldraw[black](-1.75, 1.25) circle (2pt);
        
        
        \draw[black] (1.75, -1) -- (2.25, -0.5);
        \draw[black] (2.25, -1) -- (1.75, -0.5);
        \filldraw[black](2, -0.75) circle (2pt);
        
        

    \end{scope}

\end{tikzpicture}
\]

\item Алгоритм продолжается до тех пор, пока кластеры будут изменяться. 


\end{enumerate}



\begin{tikzpicture}
  \begin{scope}[xshift=7.5cm]
        \draw (-3.5, 3.5 ) -- (3.5, 3.5);
        \draw (-3.5, -3.5) -- (3.5, -3.5);
        \draw (-3.5, -3.5) -- (-3.5, 3.5);
        \draw (3.5, 3.5) -- (3.5, -3.5);
    
        \draw[yellow, fill=yellow] (-1.5, -2.5) circle (0.2);
        \draw[red, fill=red] (-1.75, -0.5) circle (0.2);
        \draw[red, fill=red] (-1.5, 1.4) circle (0.2);
        \draw[red, fill=red] (-2, 0.75) circle (0.2);
        \draw[red, fill=red] (-2.625, 1.8) circle (0.2);
        \draw[yellow, fill=yellow] (1.75, -3.2) circle (0.2);
        \draw[yellow, fill=yellow] (3, -2.8) circle (0.2);
        \draw[yellow, fill=yellow] (3, -0.5) circle (0.2);
        \draw[yellow, fill=yellow] (1.75, -0.5) circle (0.2);
        \draw[yellow, fill=yellow] (2.7, 1.2) circle (0.2);
        \draw[yellow, fill=yellow] (2.5, 1.8) circle (0.2);
        \draw[red, fill=red] (0.5, 1) circle (0.2);
        \draw[red, fill=red] (-0.4, 2.3) circle (0.2);
        \draw[red, fill=red] (-0.9, 2.9) circle (0.2); 

        \draw[blue] (3, 3.4) -- (-2.8, -3.4);

        \draw[black] (-1, 1.5) -- (-0.5, 2);
        \draw[black] (-0.5, 1.5) -- (-1, 2);
        \filldraw[black](-0.75, 1.75) circle (2pt);
        
        
        \draw[black] (1.5, -1.5) -- (2, -1);
        \draw[black] (2, -1.5) -- (1.5, -1);
        \filldraw[black](1.75, -1.25) circle (2pt);

        
        \end{scope}
        
 \begin{scope}
        \draw (-3.5, 3.5 ) -- (3.5, 3.5);
        \draw (-3.5, -3.5) -- (3.5, -3.5);
        \draw (-3.5, -3.5) -- (-3.5, 3.5);
        \draw (3.5, 3.5) -- (3.5, -3.5);
    
        \draw[yellow, fill=yellow] (-1.5, -2.5) circle (0.2);
        \draw[red, fill=red] (-1.75, -0.5) circle (0.2);
        \draw[red, fill=red] (-1.5, 1.4) circle (0.2);
        \draw[red, fill=red] (-2, 0.75) circle (0.2);
        \draw[red, fill=red] (-2.625, 1.8) circle (0.2);
        \draw[yellow, fill=yellow] (1.75, -3.2) circle (0.2);
        \draw[yellow, fill=yellow] (3, -2.8) circle (0.2);
        \draw[yellow, fill=yellow] (3, -0.5) circle (0.2);
        \draw[yellow, fill=yellow] (1.75, -0.5) circle (0.2);
        \draw[yellow, fill=yellow] (2.7, 1.2) circle (0.2);
        \draw[yellow, fill=yellow] (2.5, 1.8) circle (0.2);
        \draw[red, fill=red] (0.5, 1) circle (0.2);
        \draw[red, fill=red] (-0.4, 2.3) circle (0.2);
        \draw[red, fill=red] (-0.9, 2.9) circle (0.2); 

        \draw[->][black] (-1.4, 1.2) -- (-0.85, 1.4);
        \draw[->][black] (1.9, -0.95) -- (1.75, -1.1);
        \draw[blue] (3, 3.4) -- (-2.5, -3.4);
                
        \draw[black] (-1, 1.5) -- (-0.5, 2);
        \draw[black] (-0.5, 1.5) -- (-1, 2);
        \filldraw[black](-0.75, 1.75) circle (2pt);
        
        
        \draw[black] (1.5, -1.5) -- (2, -1);
        \draw[black] (2, -1.5) -- (1.5, -1);
        \filldraw[black](1.75, -1.25) circle (2pt);
        
        \draw[black] (-2, 1) -- (-1.5, 1.5);
        \draw[black] (-1.5, 1) -- (-2, 1.5);
        \filldraw[black](-1.75, 1.25) circle (2pt);
        
        
        \draw[black] (1.75, -1) -- (2.25, -0.5);
        \draw[black] (2.25, -1) -- (1.75, -0.5);
        \filldraw[black](2, -0.75) circle (2pt);
    \end{scope}

  
\end{tikzpicture}





%\end{document}

\section*{Иерархическая кластеризация}


Иерархическая кластеризация выводит иерархию, структуру, которая в целом информативнее, чем набор предсказаний от других видов кластеризации. Иерархическая кластеризация не требует заранее определять количество кластеров, и большинство популярных иерархических алгоритмов является неслучайными. 

Существует два подхода решения задачи иерархической кластеризации — восходящий и нисходящий. Восходящий алгоритм присваивает каждому образцу отдельный кластер, 
а затем объединяет пары наиболее похожих друг на друга кластеров в один, пока не получит кластер со всеми объектами внутри. Нисходящий алгоритм наоборот: 
начинает с большого кластера и далее разделяет кластеры до тех пор, пока в каждом кластере не будет по одному объекту.
Часто иерархическую кластеризацию представляют в форме дендрограмм, 
где схожие объекты содержатся в одинаковых ветвях. Для перевода результата иерархической кластеризации в готовый вид необходимо сделать разрез на определенном уровне. Существует несколько правил для выбора уровня разреза:

\begin{itemize}
\item Выполнить разрез на данном уровне схожести.
\item Сделать разрез там, где разница между двумя уровнями схожести будет максимальной.
\item Зафиксировать строгое количество $K$ кластеров и выбрать соответствующий уровень для разделения.
\end{itemize}
   
\section*{Восходящая кластеризация}
Для начала стоит ввести такое понятие, как расстояние между точками. Существует несколько способов определения расстояний, но наиболее популярно евклидово расстояние, которое и будет далее использоваться:

\[
\||a-b||_2 = \sqrt{\sum_i (a_i-b_i)^2}
\]

Мы будем использовать полносвязную кластеризацию для определения схожести кластеров:

\begin{itemize}
\item В этом подходе схожесть двух кластеров определяется как схожесть их двух наиболее различающихся элементов:
\[
\ d(W_i, W_j) =  \max_{x_i \in W_i, x_j \in W_j} ||x_i - x_j||
\]
где d - некоторая функция схожести, $W_{i, j}$ - два кластера и $x_{i, j}$ - две точки, взятые из каждого кластера. При таком подходе уделяется внимание лишь тем областям, в которых кластеры наиболее близки; игнорируются другие части и общая структура.
Данный подход позволяет создавать компактные кластеры; чувствителен к выбросам.
\end{itemize}

\section*{Пошаговая иллюстрация}

\begin{enumerate}
\item Сначала алгоритм присваивает каждому объекту кластер

\[
\begin{tikzpicture}
    
    \draw (-3.5, 3.5) -- (3.5, 3.5);
    \draw (-3.5, -3.5) -- (3.5, -3.5);
    \draw (-3.5, -3.5) -- (-3.5, 3.5);
    \draw (3.5, 3.5) -- (3.5, -3.5);
        
    \draw (2.5, 2.5) circle (0.12);
    \draw (1.35, 2.25) circle (0.12);
    \draw (1.6, 1.1) circle (0.12);
    \draw (1.95, 1.9) circle (0.12);
    \draw (-2.4, -1.9) circle (0.12);
    \draw (-1.8, -1.3) circle (0.12);
    \draw (-1.95, 2.11) circle (0.12);
    \draw (2.6, -2.4) circle (0.12);
        
    \draw (-2.6, -2.15) rectangle (-2.15, -1.64);
    \draw (-2.05, -1.55) rectangle (-1.55, -1.1);
    \draw[green] (2.2, -2.8) rectangle (3, -2);
    \draw (2.2, 2.2) rectangle (2.8, 2.8);
    \draw[red] (1.023, 1.9) rectangle (1.63, 2.57);
    \draw (1.73, 1.64) rectangle (2.17, 2.13);
    \draw (1.35, 0.83) rectangle (1.85, 1.38);
    \draw[green] (-2.4, 1.6) rectangle (-1.5, 2.55);

\end{tikzpicture}
\]

\item Затем он подсчитывает схожесть (через расстояние) с соседними объектами, и в случае положительного результата объединяется с ближайшими объектами. На этом уровне кластеризации уже можно сделать разрез.

\[
\begin{tikzpicture}
    \draw (-3.5, 3.5) -- (3.5, 3.5);
    \draw (-3.5, -3.5) -- (3.5, -3.5);
    \draw (-3.5, -3.5) -- (-3.5, 3.5);
    \draw (3.5, 3.5) -- (3.5, -3.5);
        
    \draw (2.5, 2.5) circle (0.12);
    \draw (1.35, 2.25) circle (0.12);
    \draw (1.6, 1.1) circle (0.12);
    \draw (1.95, 1.9) circle (0.12);
    \draw (-2.4, -1.9) circle (0.12);
    \draw (-1.8, -1.3) circle (0.12);
    \draw (-1.95, 2.11) circle (0.12);
    \draw (2.6, -2.4) circle (0.12);
        
    \draw[blue] (-2.7, -2.1) rectangle (-1.5, -1.1);
    \draw[green] (2.2, -2.8) rectangle (3, -2);
    \draw[red] (1, 0.83) rectangle (2.8, 2.8);
    \draw[green] (-2.4, 1.6) rectangle (-1.5, 2.55);
\end{tikzpicture}
\]

\item Алгоритм повторяет этот шаг, пока не останется других кластеров. Ближайший объект присоединяет к себе. Здесь он уже захватил отметку из шума.

\[
\begin{tikzpicture}
\draw (-3.5, 3.5) -- (3.5, 3.5);
    \draw (-3.5, -3.5) -- (3.5, -3.5);
    \draw (-3.5, -3.5) -- (-3.5, 3.5);
    \draw (3.5, 3.5) -- (3.5, -3.5);
        
    \draw (2.5, 2.5) circle (0.12);
    \draw (1.35, 2.25) circle (0.12);
    \draw (1.6, 1.1) circle (0.12);
    \draw (1.95, 1.9) circle (0.12);
    \draw (-2.4, -1.9) circle (0.12);
    \draw (-1.8, -1.3) circle (0.12);
    \draw (-1.95, 2.11) circle (0.12);
    \draw (2.6, -2.4) circle (0.12);
        
    \draw[red] (-2.3, 0.83) rectangle (2.8, 2.8);
    \draw[blue] (-2.7, -2.1) rectangle (-1.5, -1.1);
    \draw[green] (2.2, -2.8) rectangle (3, -2);
\end{tikzpicture}
\]

\item Наконец, алгоритм захватывает все точки на плоскости, объединяя их в один единый кластер.

\[
\begin{tikzpicture}
\draw (-3.5, 3.5) -- (3.5, 3.5);
    \draw (-3.5, -3.5) -- (3.5, -3.5);
    \draw (-3.5, -3.5) -- (-3.5, 3.5);
    \draw (3.5, 3.5) -- (3.5, -3.5);
        
    \draw (2.5, 2.5) circle (0.12);
    \draw (1.35, 2.25) circle (0.12);
    \draw (1.6, 1.1) circle (0.12);
    \draw (1.95, 1.9) circle (0.12);
    \draw (-2.4, -1.9) circle (0.12);
    \draw (-1.8, -1.3) circle (0.12);
    \draw (-1.95, 2.11) circle (0.12);
    \draw (2.6, -2.4) circle (0.12);
        
    \draw[red] (-2.7, -2.7) rectangle (2.8, 2.85);
\end{tikzpicture}
\]
\end{enumerate}

\newpage
\section*{Нисходящая кластеризация}
Алгоритм выполняет следующие шаги:

\begin{enumerate}
\item На первом шаге алгоритм присваивает всем объектам один кластер.
\[
\begin{tikzpicture}
        \draw (-3.5, 3.5) -- (3.5, 3.5);
        \draw (-3.5, -3.5) -- (3.5, -3.5);
        \draw (-3.5, -3.5) -- (-3.5, 3.5);
        \draw (3.5, 3.5) -- (3.5, -3.5);
        
        \draw (-2.8, -2.8) circle (0.12);
        \draw (-1.75, 2.8) circle (0.12);
        \draw (1, 2.5) circle (0.12);
        \draw (1.9, 1) circle (0.12);
        \draw (1, -1.5) circle (0.12);
        \draw (1.8, -2) circle (0.12);

        \draw[red] (-3, -3) rectangle (3, 3);
\end{tikzpicture}
\]

\item Далее алгоритм находит наименее похожий на другие объект и отделяет его, создав новый кластер. 

\[
\begin{tikzpicture}

        \begin{scope}
        \draw (-3.5, 3.5) -- (3.5, 3.5);
        \draw (-3.5, -3.5) -- (3.5, -3.5);
        \draw (-3.5, -3.5) -- (-3.5, 3.5);
        \draw (3.5, 3.5) -- (3.5, -3.5);

        \draw (-2.8, -2.8) circle (0.12);
        \draw (-1.75, 2.8) circle (0.12);
        \draw (1, 2.5) circle (0.12);
        \draw (1.9, 1) circle (0.12);
        \draw (1, -1.5) circle (0.12);
        \draw (1.8, -2) circle (0.12);

        
        \draw[red] (-2.3, -2.3) rectangle (3, 3);
        \draw[blue] (-3, -3) rectangle (-2.6, -2.6);
    \end{scope}
\end{tikzpicture}
\]

\item В исходном кластере алгоритм рассчитывает среднее расстояние для каждого объекта со всеми остальными. Затем рассчитывает расстояние для каждого объектами и всеми объектами, которые находятся в  отдельном кластере. Если первое расстояние больше, то объект переносится во второй кластер. 

\[
\begin{tikzpicture}
        \begin{scope}[yshift=-9cm]
        \draw (-3.5, 3.5) -- (3.5, 3.5);
        \draw (-3.5, -3.5) -- (3.5, -3.5);
        \draw (-3.5, -3.5) -- (-3.5, 3.5);
        \draw (3.5, 3.5) -- (3.5, -3.5);

        \draw (-2.8, -2.8) circle (0.12);
        \draw (-1.75, 2.8) circle (0.12);
        \draw (1, 2.5) circle (0.12);
        \draw (1.9, 1) circle (0.12);
        \draw (1, -1.5) circle (0.12);
        \draw (1.8, -2) circle (0.12);

        
        \draw[blue] (-2.5, 0.5) rectangle (3, 3);
        \draw[green] (-3, -3) rectangle (2.8, -1);
    \end{scope}
\end{tikzpicture}
\]

\item Алгоритм повторяет шаг 3., пока объектов с отрицательной разностью не останется.

\item Далее алгоритм повторяет шаги 2-4 для самых больших кластеров. 

\[
\begin{tikzpicture}
        \begin{scope}[yshift=-9cm]
        \draw (-3.5, 3.5) -- (3.5, 3.5);
        \draw (-3.5, -3.5) -- (3.5, -3.5);
        \draw (-3.5, -3.5) -- (-3.5, 3.5);
        \draw (3.5, 3.5) -- (3.5, -3.5);

        \draw (-2.8, -2.8) circle (0.12);
        \draw (-1.75, 2.8) circle (0.12);
        \draw (1, 2.5) circle (0.12);
        \draw (1.9, 1) circle (0.12);
        \draw (1, -1.5) circle (0.12);
        \draw (1.8, -2) circle (0.12);

        
        \draw[blue] (-2, 3) rectangle (-1.5, 2.5);
        \draw[green] (0.5, 3) rectangle (2.8, 0);
        \draw[black] (0.5, -1) rectangle (2.5, -2.5);
        \draw[red] (-3, -3) rectangle (-2.6, -2.6);
    \end{scope}
\end{tikzpicture}
\]


\item Алгоритм повторяет шаги 2-5,  пока каждому образцу не будет соответствовать свой отдельный кластер.

\[
\begin{tikzpicture}
        \begin{scope}[yshift=-9cm]
        \draw (-3.5, 3.5) -- (3.5, 3.5);
        \draw (-3.5, -3.5) -- (3.5, -3.5);
        \draw (-3.5, -3.5) -- (-3.5, 3.5);
        \draw (3.5, 3.5) -- (3.5, -3.5);

        \draw (-2.8, -2.8) circle (0.12);
        \draw (-1.75, 2.8) circle (0.12);
        \draw (1, 2.5) circle (0.12);
        \draw (1.9, 1) circle (0.12);
        \draw (1, -1.5) circle (0.12);
        \draw (1.8, -2) circle (0.12);

        
        \draw[blue] (-1.95, 3) rectangle (-1.55, 2.6);
        \draw[red] (-3, -3) rectangle (-2.6, -2.6);
        \draw[green] (0.8, 2.7) rectangle (1.2, 2.3);
        \draw[black] (1.7, 1.2) rectangle (2.1, 0.8);
        \draw[magenta] (0.8, -1.3) rectangle (1.2, -1.7);
        \draw[cyan] (1.6, -1.8) rectangle (2, -2.2);
    \end{scope}
\end{tikzpicture}
\]

\end{enumerate}

Подсчитать среднюю схожесть образцов в исходном кластере с остальными объектами этого кластера и вычесть из нее схожесть данного объекта с образцами в новом кластере. Если эта величина окажется отрицательной, перенести
этот объект в новый кластер.
\[
\begin{tikzpicture}
\node (a) at (-6,0) {a};
\node (b) at (-3,0) {b};
\node (c) at (-0.5,0) {c};
\node (d) at (0.5,0) {d};
\node (e) at (1.5,0) {e};
\node (f) at (2.5,0) {f};

\node[blue] (ab) at (-4.5,2.5) {};
\node (cd) at (0,1) {};
\node (ef) at (2,1) {};
\node (cdef) at (1,2.5) {};
\node (all) at (-1.5,5) {};

\draw[blue]  (a) |- (ab.center);
\draw[blue]  (b) |- (ab.center);
\draw[red]  (c) |- (cd.center);
\draw[red]  (d) |- (cd.center);
\draw[red]  (e) |- (ef.center);
\draw[red]  (f) |- (ef.center);

\draw[red]  (cd.center) |- (cdef.center);
\draw[red]  (ef.center) |- (cdef.center);
\draw  (ab.center) |- (all.center);
\draw  (cdef.center) |- (all.center);
\end{tikzpicture}
\]

%\documentclass[12pt, a4paper]{article}

%\usepackage{cmap}					
%\usepackage[T2A]{fontenc}		
%\usepackage[utf8]{inputenc}			
%\usepackage[english, russian]{babel}
%\usepackage{amsmath}
%\usepackage{tikz}

%\begin{document}

\section*{DBSCAN}

Алгоритм кластеризации DBSCAN – самый молодой алгоритм среди представленных. Он позволяет находить кластеры произвольной формы в пространстве.

Суть работы алгоритма заключается в <<захвате>> кластеров через случайно найденную точку. Если у найденной точки есть соседи, то он их «заражает», и уже от них продолжает поиск новых соседей. Так продолжается до тех пор, пока не будет <<заражена>> последняя точка в предполагаемом кластере. Если количество зараженных объектов совпадает с требованием, то они превращаются в кластер. В ином случае они определяются как выбросы, и им присваивается соответствующая метка.

Соответственно, на вход в алгоритм подается максимальное расстояние для <<заражения>> и минимальное количество уже <<зараженных>> точек для формирования кластера. Далее DBSCAN действует сам.

Сложность его работы определяется количеством  объектов. Для экономии времени часто используют смешанный DBSCAN с K-means алгоритмом, реже смешанные с другими алгоритмами.

\section*{Пошаговая иллюстрация работы}

\[
\begin{tikzpicture}
    \draw (-3.5, 3.5) -- (3.5, 3.5);
    \draw (-3.5, -3.5) -- (3.5, -3.5);
    \draw (-3.5, -3.5) -- (-3.5, 3.5);
    \draw (3.5, 3.5) -- (3.5, -3.5);
        
    \draw[red, fill=red] (2.5, 2.5) circle (0.12);
    \draw (1.35, 2.25) circle (0.12);
    \draw (1.6, 1.1) circle (0.12);
    \draw (1.95, 1.9) circle (0.12);
    \draw (-2.4, -1.9) circle (0.12);
    \draw (-1.8, -1.3) circle (0.12);
    \draw (-1.95, 2.11) circle (0.12);
    \draw (2.6, -2.4) circle (0.12);
        
    \draw[->][red] (2.435, 2.435) -- (2.04, 2.01);
    \draw[->][red] (1.88, 1.77) -- (1.64, 1.23);
    \draw[->][red] (1.82, 1.96) -- (1.47, 2.18);
\end{tikzpicture}
\]
\begin{enumerate}
\item DBSCAN нашел случайную точку в пространстве и <<заразил>> ее. Далее происходит последовательный захват соседей, находящихся друг от друга в пределах требуемого расстояния. 

\[
\begin{tikzpicture}
    \draw (-3.5, 3.5) -- (3.5, 3.5);
    \draw (-3.5, -3.5) -- (3.5, -3.5);
    \draw (-3.5, -3.5) -- (-3.5, 3.5);
    \draw (3.5, 3.5) -- (3.5, -3.5);
        
    \draw[red, fill=red] (2.5, 2.5) circle (0.12);
    \draw[red, fill=red] (1.35, 2.25) circle (0.12);
    \draw[red, fill=red] (1.6, 1.1) circle (0.12);
    \draw[red, fill=red] (1.95, 1.9) circle (0.12);
    \draw (-2.4, -1.9) circle (0.12);
    \draw (-1.8, -1.3) circle (0.12);
    \draw[green, fill=green] (-1.95, 2.11) circle (0.12);
    \draw (2.6, -2.4) circle (0.12);
        
    \draw[<->][red] (2.435, 2.435) -- (2.04, 2.01);
    \draw[<->][red] (1.88, 1.77) -- (1.64, 1.23);
    \draw[<->][red] (1.82, 1.96) -- (1.47, 2.18);
    \draw[<->][red] (1.54, 1.24) -- (1.30, 2.13);
    \draw[<->][red] (1.74, 1.19) -- (2.5, 2.36);
    \draw[<->][red] (1.47, 2.27) -- (2.37, 2.5);
\end{tikzpicture}
\]
\item Алгоритм понял, что все цели поражены, и на захваченных территориях DBSCAN принимает решение создать тоталитарное государство-кластер. Но и этого ему мало. Он возобновляет поиск. К счастью, пока на пути безжалостной машины попался лишь выброс.

\[
\begin{tikzpicture}
        \draw (-3.5, 3.5) -- (3.5, 3.5);
        \draw (-3.5, -3.5) -- (3.5, -3.5);
        \draw (-3.5, -3.5) -- (-3.5, 3.5);
        \draw (3.5, 3.5) -- (3.5, -3.5);
        
        \draw[red, fill=red] (2.5, 2.5) circle (0.12);
        \draw[red, fill=red] (1.35, 2.25) circle (0.12);
        \draw[red, fill=red] (1.6, 1.1) circle (0.12);
        \draw[red, fill=red] (1.95, 1.9) circle (0.12);
        \draw (-2.4, -1.9) circle (0.12);
        \draw[blue, fill=blue] (-1.8, -1.3) circle (0.12);
        \draw[green, fill=green] (-1.95, 2.11) circle (0.12);
        \draw (2.6, -2.4) circle (0.12);
        
        \draw[->][red] (2.435, 2.435) -- (2.04, 2.01);
        \draw[->][red] (1.88, 1.77) -- (1.64, 1.23);
        \draw[->][red] (1.82, 1.96) -- (1.47, 2.18);
        \draw[<->][red] (1.54, 1.24) -- (1.30, 2.13);
        \draw[<->][red] (1.74, 1.19) -- (2.5, 2.36);
        \draw[<->][red] (1.47, 2.27) -- (2.37, 2.5);
        
        \draw[->][blue] (-1.8, -1.3) -- (-2.32, -1.79);
\end{tikzpicture}
\]
\item Найдя точку с соседом рядом , алгоритм захватывает эти два объекта и присваивает им метку кластера. Снова запускается поиск.

\[
\begin{tikzpicture}
        \draw (-3.5, 3.5) -- (3.5, 3.5);
        \draw (-3.5, -3.5) -- (3.5, -3.5);
        \draw (-3.5, -3.5) -- (-3.5, 3.5);
        \draw (3.5, 3.5) -- (3.5, -3.5);
        
        \draw[red, fill=red] (2.5, 2.5) circle (0.12);
        \draw[red, fill=red] (1.35, 2.25) circle (0.12);
        \draw[red, fill=red] (1.6, 1.1) circle (0.12);
        \draw[red, fill=red] (1.95, 1.9) circle (0.12);
        \draw[blue, fill=blue] (-2.4, -1.9) circle (0.12);
        \draw[blue, fill=blue] (-1.8, -1.3) circle (0.12);
        \draw[green, fill=green] (-1.95, 2.11) circle (0.12);
        \draw[green, fill=green] (2.6, -2.4) circle (0.12);
        
        \draw[->][red] (2.435, 2.435) -- (2.04, 2.01);
        \draw[->][red] (1.88, 1.77) -- (1.64, 1.23);
        \draw[->][red] (1.82, 1.96) -- (1.47, 2.18);
        \draw[<->][red] (1.54, 1.24) -- (1.30, 2.13);
        \draw[<->][red] (1.74, 1.19) -- (2.5, 2.36);
        \draw[<->][red] (1.47, 2.27) -- (2.37, 2.5);
        
        \draw[<->][blue] (-1.93, -1.38) -- (-2.32, -1.79);
\end{tikzpicture}
\]
\item К сожалению для алгоритма, последний найденный объект является выбросом. Не найдя соседей, он заканчивает свою работу.
\end{enumerate}
%\end{document}



\end{document}
