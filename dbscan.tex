%\documentclass[12pt, a4paper]{article}

%\usepackage{cmap}					
%\usepackage[T2A]{fontenc}		
%\usepackage[utf8]{inputenc}			
%\usepackage[english, russian]{babel}
%\usepackage{amsmath}
%\usepackage{tikz}

%\begin{document}

\section*{DBSCAN}

Алгоритм кластеризации DBSCAN – самый молодой алгоритм среди представленных. Он позволяет находить кластеры произвольной формы в пространстве.

Суть работы алгоритма заключается в <<захвате>> кластеров через случайно найденную точку. Если у найденной точки есть соседи, то он их «заражает», и уже от них продолжает поиск новых соседей. Так продолжается до тех пор, пока не будет <<заражена>> последняя точка в предполагаемом кластере. Если количество зараженных объектов совпадает с требованием, то они превращаются в кластер. В ином случае они определяются как выбросы, и им присваивается соответствующая метка.

Соответственно, на вход в алгоритм подается максимальное расстояние для <<заражения>> и минимальное количество уже <<зараженных>> точек для формирования кластера. Далее DBSCAN действует сам.

Сложность его работы определяется количеством  объектов. Для экономии времени часто используют смешанный DBSCAN с K-means алгоритмом, реже смешанные с другими алгоритмами.

\section*{Пошаговая иллюстрация работы}

\[
\begin{tikzpicture}
    \draw (-3.5, 3.5) -- (3.5, 3.5);
    \draw (-3.5, -3.5) -- (3.5, -3.5);
    \draw (-3.5, -3.5) -- (-3.5, 3.5);
    \draw (3.5, 3.5) -- (3.5, -3.5);
        
    \draw[red, fill=red] (2.5, 2.5) circle (0.12);
    \draw (1.35, 2.25) circle (0.12);
    \draw (1.6, 1.1) circle (0.12);
    \draw (1.95, 1.9) circle (0.12);
    \draw (-2.4, -1.9) circle (0.12);
    \draw (-1.8, -1.3) circle (0.12);
    \draw (-1.95, 2.11) circle (0.12);
    \draw (2.6, -2.4) circle (0.12);
        
    \draw[->][red] (2.435, 2.435) -- (2.04, 2.01);
    \draw[->][red] (1.88, 1.77) -- (1.64, 1.23);
    \draw[->][red] (1.82, 1.96) -- (1.47, 2.18);
\end{tikzpicture}
\]
\begin{enumerate}
\item DBSCAN нашел случайную точку в пространстве и <<заразил>> ее. Далее происходит последовательный захват соседей, находящихся друг от друга в пределах требуемого расстояния. 

\[
\begin{tikzpicture}
    \draw (-3.5, 3.5) -- (3.5, 3.5);
    \draw (-3.5, -3.5) -- (3.5, -3.5);
    \draw (-3.5, -3.5) -- (-3.5, 3.5);
    \draw (3.5, 3.5) -- (3.5, -3.5);
        
    \draw[red, fill=red] (2.5, 2.5) circle (0.12);
    \draw[red, fill=red] (1.35, 2.25) circle (0.12);
    \draw[red, fill=red] (1.6, 1.1) circle (0.12);
    \draw[red, fill=red] (1.95, 1.9) circle (0.12);
    \draw (-2.4, -1.9) circle (0.12);
    \draw (-1.8, -1.3) circle (0.12);
    \draw[green, fill=green] (-1.95, 2.11) circle (0.12);
    \draw (2.6, -2.4) circle (0.12);
        
    \draw[<->][red] (2.435, 2.435) -- (2.04, 2.01);
    \draw[<->][red] (1.88, 1.77) -- (1.64, 1.23);
    \draw[<->][red] (1.82, 1.96) -- (1.47, 2.18);
    \draw[<->][red] (1.54, 1.24) -- (1.30, 2.13);
    \draw[<->][red] (1.74, 1.19) -- (2.5, 2.36);
    \draw[<->][red] (1.47, 2.27) -- (2.37, 2.5);
\end{tikzpicture}
\]
\item Алгоритм понял, что все цели поражены, и на захваченных территориях DBSCAN принимает решение создать тоталитарное государство-кластер. Но и этого ему мало. Он возобновляет поиск. К счастью, пока на пути безжалостной машины попался лишь выброс.

\[
\begin{tikzpicture}
        \draw (-3.5, 3.5) -- (3.5, 3.5);
        \draw (-3.5, -3.5) -- (3.5, -3.5);
        \draw (-3.5, -3.5) -- (-3.5, 3.5);
        \draw (3.5, 3.5) -- (3.5, -3.5);
        
        \draw[red, fill=red] (2.5, 2.5) circle (0.12);
        \draw[red, fill=red] (1.35, 2.25) circle (0.12);
        \draw[red, fill=red] (1.6, 1.1) circle (0.12);
        \draw[red, fill=red] (1.95, 1.9) circle (0.12);
        \draw (-2.4, -1.9) circle (0.12);
        \draw[blue, fill=blue] (-1.8, -1.3) circle (0.12);
        \draw[green, fill=green] (-1.95, 2.11) circle (0.12);
        \draw (2.6, -2.4) circle (0.12);
        
        \draw[->][red] (2.435, 2.435) -- (2.04, 2.01);
        \draw[->][red] (1.88, 1.77) -- (1.64, 1.23);
        \draw[->][red] (1.82, 1.96) -- (1.47, 2.18);
        \draw[<->][red] (1.54, 1.24) -- (1.30, 2.13);
        \draw[<->][red] (1.74, 1.19) -- (2.5, 2.36);
        \draw[<->][red] (1.47, 2.27) -- (2.37, 2.5);
        
        \draw[->][blue] (-1.8, -1.3) -- (-2.32, -1.79);
\end{tikzpicture}
\]
\item Найдя точку с соседом рядом , алгоритм захватывает эти два объекта и присваивает им метку кластера. Снова запускается поиск.

\[
\begin{tikzpicture}
        \draw (-3.5, 3.5) -- (3.5, 3.5);
        \draw (-3.5, -3.5) -- (3.5, -3.5);
        \draw (-3.5, -3.5) -- (-3.5, 3.5);
        \draw (3.5, 3.5) -- (3.5, -3.5);
        
        \draw[red, fill=red] (2.5, 2.5) circle (0.12);
        \draw[red, fill=red] (1.35, 2.25) circle (0.12);
        \draw[red, fill=red] (1.6, 1.1) circle (0.12);
        \draw[red, fill=red] (1.95, 1.9) circle (0.12);
        \draw[blue, fill=blue] (-2.4, -1.9) circle (0.12);
        \draw[blue, fill=blue] (-1.8, -1.3) circle (0.12);
        \draw[green, fill=green] (-1.95, 2.11) circle (0.12);
        \draw[green, fill=green] (2.6, -2.4) circle (0.12);
        
        \draw[->][red] (2.435, 2.435) -- (2.04, 2.01);
        \draw[->][red] (1.88, 1.77) -- (1.64, 1.23);
        \draw[->][red] (1.82, 1.96) -- (1.47, 2.18);
        \draw[<->][red] (1.54, 1.24) -- (1.30, 2.13);
        \draw[<->][red] (1.74, 1.19) -- (2.5, 2.36);
        \draw[<->][red] (1.47, 2.27) -- (2.37, 2.5);
        
        \draw[<->][blue] (-1.93, -1.38) -- (-2.32, -1.79);
\end{tikzpicture}
\]
\item К сожалению для алгоритма, последний найденный объект является выбросом. Не найдя соседей, он заканчивает свою работу.
\end{enumerate}
%\end{document}

